\documentclass[a4paper, 10pt, authoryear]{elsarticle}

\makeatletter
\def\ps@pprintTitle{%
  \let\@oddhead\relax
  \let\@evenhead\relax
  \let\@oddfoot\relax
  \let\@evenfoot\relax
}
\makeatother

\usepackage{times}
\usepackage{amsmath}
\usepackage{amssymb}
\usepackage{graphicx}
\usepackage{subfigure}
\usepackage{color}
\usepackage{multirow}
\usepackage{url}
\usepackage{hyperref}
\usepackage{float}
\usepackage{bookmark}
\usepackage{booktabs}
\usepackage{longtable}
\usepackage{tabularx}
\usepackage{rotating}
\usepackage{lscape}
\usepackage{pdflscape}
\usepackage{geometry}
\usepackage{caption}
\usepackage{setspace}
\usepackage{lineno}
%\usepackage[backend=bibtex, style=authoryear, natbib=true]{biblatex} 
%\addbibresource{proposal1.bib}
%\usepackage[authoryear]{natbib}
\bibliographystyle{elsarticle-harv}
\geometry{left=2.5cm,right=2.5cm,top=2.5cm,bottom=2.5cm}

\begin{document}
\title{
    Human Capital Risk \\
    \large FIN9013 Empirical Corporate Finance Research Proposal
}
\author{Ang Zhang}

% \begin{abstract}
%     This proposal outlines a research plan for investigating the impact of corporate finance strategies on firm performance. The study will employ quantitative methods to analyze financial data from a sample of companies over a ten-year period. Key areas of focus include capital structure, dividend policy, and investment decisions. The findings are expected to provide insights into how financial management practices influence profitability and growth.
% \end{abstract}

\maketitle

\section{Introduction}

"Not everything that can be counted counts, and not everything that counts can be counted." - William Bruce Cameron (1958)
\newline
Organizational capital is a crucial component of firm assets that do not appear on the balance sheet. Some research, such as \cite{iqbalBetterEstimateInternally2024}, has attempted to provide better estimates of intangible capital for financial statement users. \cite{petersIntangibleCapitalInvestmentq2017} proposed a new Q measure that explicitly accounts for organizational capital in a firm's performance. Although intangible, the returns generated from organizational capital are tangible and appreciated by investors through firm valuation. Just as not all balance sheet assets produce the same ROA, not all organizational capital yields equal value or shares the same risk profile. Consequently, investors explicitly price organizational capital based on their assessment of its quality.
\newline
Management studies like \cite{campbellWhoLeavesWhere2012} have shown that employee mobility negatively impacts performance in knowledge-intensive sectors. The investment community has also recognized the importance of human capital risk and requested expanded human capital disclosure in their petition to the SEC. Studies like \cite{bennedsenCEOsMatterEvidence2020} have measured the contribution of C-suite executives to firm performance, while others like \cite{liEmployeeTurnoverFirm2022} have used employee turnover as a human capital metric, discovering significant associations with firm performance metrics. However, attrition can be a noisy measure of human capital risk and is not easily comparable across industries. Better measures of human capital risk are needed.
\newline
Firm disclosure and investor communications help reduce information asymmetry. Researchers have used textual analysis to compile various firm-level risks from regulatory filings. For instance, \cite{florackisCybersecurityRisk2023} compiled cybersecurity risks from 10-K and 10-Q filings, showing that these risks are priced by the market. \cite{sautnerFirmLevelClimateChange2023} used textual analysis of earnings call transcripts to identify climate change risk, which was also shown to be priced. Textual analysis enables researchers to quantify previously unquantifiable risks.

\section{Methodologies}
The study will compile a human capital risk measure from textual analysis and examine its relationship with firm performance measures. Textual data will be sourced from firm filings and earnings call transcripts, and text embedding will be used to construct a human capital risk exposure. An industry-level and economy-level human capital shock index will be constructed by analyzing a textual dataset from news articles. Human capital factor portfolios will be constructed based on human capital risk exposure, and their interaction with human capital shocks will be examined. The study will also explore the relationship between human capital risk and firm performance measures.
\newline
To examine the response of this risk metric to events, the study will use a short-run event study to analyze the SR CAR of firms experiencing positive and negative human capital events. Positive events include the appointment of new C-suite members or key employees, while negative events include the resignation of C-suite members, key employees, or large-scale employee departures.

\section{Expected Outcomes}
The study aims to find how human capital risk is priced in the market. The results will provide insights into how much investors value human capital risk and how this value varies in firms with different characteristics. The findings will also inform managers about the trade-off between human capital risk and firm performance, helping them make better human capital decisions with risk in mind.

\pagebreak

\bibliography{proposal2}

\end{document}