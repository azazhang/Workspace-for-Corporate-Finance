\documentclass[12pt, times]{article}
\usepackage{booktabs}
\usepackage{makecell}
\usepackage{multirow}
\usepackage{fullpage}
\usepackage{csvsimple}
\usepackage{setspace}

\title{FIN9013 Assignment 1}
\author{Ang Zhang}

\begin{document}
\maketitle

\onehalfspacing


\section*{Linear regression, interactions, split-samples and non-linearities}

In the baseline model, the slope term of $Size$ factor is 0.0237, 
which means that for a one unit increase in the logarithm of annual sales turnover, 
measured in millions, the predicted value of the debt ratio, measured by the total debt over total book asset, \
increase by 0.0237, holding all other variables constant.
The slope term of $MB$ factor is 0.0085, which means that for a one unit increase in the market to book ratio of assets, 
the predicted value of the debt ratio, measured by the total debt over total book asset, \
increase by 0.0.0085, holding all other variables constant.
The slope term of $D(RD)$ factor is -0.0902, which means that for a firm with positive R\&D spending, 
the predicted value of the debt ratio, measured by the total debt over total book asset, \
is 0.0902 lower than a comparable firm with no R\&D spending, holding all other variables constant.
\newline
However, we can hardly say that this model is a good fit, as the $R^2$ value is only 0.0813, which means that only 8.13\% of the
variation in the dependent variable, $Debt$, can be explained by the independent variables, $Size$, $MB$ and $D(RD)$,
while the rest 91.87\% of the variation is due to other factors not included in the model.
\newline
When all variables are held at their mean, the predicted value of $Debt$ is 0.2029, while for all variables at their 
median the predicted $Debt$ value is 0.1655.
\newline
The effect of $D(RD)$ on $Debt$ is constant and always at $\beta_{rd}$, no matter where MB stands at.
That's because in eq.(1), there's no interaction term so the coefficient of $D(RD)$ which is $\beta_{rd}$ solely determines the effect
that $D(RD)$ has on $Debt$.
\newline
Comparing the coefficient of $MB$ in the RD Model and the No RD Model, 
we can see that the coefficient of $MB$ in the RD Model is 0.0015, while 
in the No RD Model it is -0.0173. This suggests that the effect of $MB$ have different
effects on the debt ratio for firms with and without R\&D spending. 
\newline
Comparing the two models where RD *rank* is included as rank in one of them (RD Rank Model)
and RD *rank* is included as class in the other one (RD Class Model), we can see that the parameter
estimates for the four categorical variables are quite different. Also, the
 RD Rank Model has a higher $R^2$ value, which means that it explains more of the 
 variation in the dependent variable. This suggests that the effect of R\&D spending
 might is likely non-linear, and that the RD Rank Model is a better fit for the data.

\section*{Discrete choice variables and censoring}

\section*{Instrumental variables}

\section*{Appendix}

\begin{table}[h!]
    \centering
    \caption{Summary Statistics Table}
    \label{table:1}
    \begin{tabular}{lcccccccc}
        \toprule
        \multicolumn{8}{c}{Panel A: All Firms} \\
        \midrule
        Variable & N & Mean & StdDev & Min & P25 & Median & P75 & Max \\
        \midrule
        Debt & 16257 & 0.197 & 0.225 & 0 & 0.000 & 0.129 & 0.317 & 1.017 \\
        Size & 16257 & 5.786 & 2.060 & -0.997 & 4.576 & 5.991 & 7.194 & 10.068 \\
        MB & 16257 & 1.912 & 1.549 & 0.334 & 0.923 & 1.397 & 2.344 & 8.834 \\
        D\_RD & 16257 & 0.582 & 0.493 & 0 & 0 & 1.000 & 1.000 & 1.000 \\
        RD\_rank & 16257 & 1.908 & 1.120 & 1.000 & 1.000 & 2.000 & 2.000 & 5.000 \\
        D\_Debt & 16257 & 0.336 & 0.472 & 0 & 0 & 0 & 1.000 & 1.000 \\
        D\_DecIPO & 16257 & 0.089 & 0.285 & 0 & 0 & 0 & 0 & 1.000 \\
        \midrule
        \multicolumn{8}{c}{Panel B: Firms with R\&D} \\
        \midrule
        Variable & N & Mean & StdDev & Min & P25 & Median & P75 & Max \\
        \midrule
        Debt & 9468 & 0.150 & 0.201 & 0 & 0 & 0.058 & 0.246 & 1.017 \\
        Size & 9468 & 5.183 & 2.168 & -0.997 & 3.908 & 5.299 & 6.658 & 10.068 \\
        MB & 9468 & 2.202 & 1.699 & 0.334 & 1.064 & 1.644 & 2.760 & 8.834 \\
        D\_RD & 9468 & 1.000 & 0 & 1.000 & 1.000 & 1.000 & 1.000 & 1.000 \\
        RD\_rank & 9468 & 2.559 & 1.068 & 2.000 & 2.000 & 2.000 & 3.000 & 5.000 \\
        D\_Debt & 9468 & 0.245 & 0.430 & 0 & 0 & 0 & 0 & 1.000 \\
        D\_DecIPO & 9468 & 0.079 & 0.270 & 0 & 0 & 0 & 0 & 1.000 \\
        \midrule
        \multicolumn{8}{c}{Panel C: Firms without R\&D} \\
        \midrule
        Variable & N & Mean & StdDev & Min & P25 & Median & P75 & Max \\
        \midrule
        Debt & 6789 & 0.264 & 0.239 & 0 & 0.049 & 0.226 & 0.401 & 1.017 \\
        Size & 6789 & 6.626 & 1.548 & -0.997 & 5.686 & 6.647 & 7.689 & 10.068 \\
        MB & 6789 & 1.508 & 1.199 & 0.334 & 0.807 & 1.140 & 1.748 & 8.834 \\
        D\_RD & 6789 & 0 & 0 & 0 & 0 & 0 & 0 & 0 \\
        RD\_rank & 6789 & 1.000 & 0 & 1.000 & 1.000 & 1.000 & 1.000 & 1.000 \\
        D\_Debt & 6789 & 0.463 & 0.499 & 0 & 0 & 0 & 1.000 & 1.000 \\
        D\_DecIPO & 6789 & 0.103 & 0.304 & 0 & 0 & 0 & 0 & 1.000 \\
        \bottomrule
    \end{tabular}
    \newline
    \textit{Note: This table presents the summary statistics for the variables used in the analysis. Panel A includes all firms, Panel B includes firms with R\&D, and Panel C includes firms without R\&D.}
\end{table}

\begin{table}[h!]
    \centering
    \caption{Parameter Estimates of Each Model}
    \label{table:2}
    \begin{tabular}{lccccccc}
        \toprule
        Parameter & \makecell{Model \\ Baseline} & \makecell{Model \\ Interaction} & \makecell{Model \\ Augmented} & \makecell{Model \\ RD} & \makecell{Model \\ No RD} & \makecell{Model \\ RD Rank} & \makecell{Model \\ RD Class} \\
        \midrule
        Intercept & \makecell{0.1114*** \\ (0.0068)} & \makecell{0.1298*** \\ (0.0072)} & \makecell{0.1029*** \\ (0.0119)} & \makecell{0.0292*** \\ (0.0063)} & \makecell{0.1029*** \\ (0.0131)} & \makecell{0.0376*** \\ (0.0090)} & \makecell{0.1070*** \\ (0.0071)} \\
        Size & \makecell{0.0237*** \\ (0.0009)} & \makecell{0.0242*** \\ (0.0009)} & \makecell{0.0282*** \\ (0.0017)} & \makecell{0.0227*** \\ (0.0009)} & \makecell{0.0282*** \\ (0.0018)} & \makecell{0.0297*** \\ (0.0010)} & \makecell{0.0334*** \\ (0.0010)} \\
        MB & \makecell{-0.0033*** \\ (0.0011)} & \makecell{-0.0175*** \\ (0.0022)} & \makecell{-0.0173*** \\ (0.0022)} & \makecell{0.0015 \\ (0.0012)} & \makecell{-0.0173*** \\ (0.0024)} & \makecell{-0.0072*** \\ (0.0012)} & \makecell{-0.0067*** \\ (0.0011)} \\
        D\_RD & \makecell{-0.0771*** \\ (0.0037)} & \makecell{-0.1092*** \\ (0.0055)} & \makecell{-0.0737*** \\ (0.0137)} & & & \\
        MB*D\_RD & & \makecell{0.0194*** \\ (0.0025)} & \makecell{0.0188*** \\ (0.0025)} & & & \\
        Size*D\_RD & & & \makecell{-0.0055*** \\ (0.0020)} & & & \\
        RD\_rank & & & & & &\makecell{0.0008 \\ (0.0020)} &  \\
        RD\_rank 1 & & & & & & & \makecell{-0.0547*** \\ (0.0081)} \\
        RD\_rank 2 & & & & & & & \makecell{-0.1358*** \\ (0.0076)} \\
        RD\_rank 3 & & & & & & & \makecell{-0.1473*** \\ (0.0097)} \\
        RD\_rank 4 & & & & & & & \makecell{-0.0337** \\ (0.0158)} \\
        \midrule
        RSquare & 0.1054 & 0.1086 & 0.1091 & 0.0586 & 0.0419 & 0.0809 & 0.1253 \\
        \#Obs & 16257 & 16257 & 16257 & 9468 & 6789 & 16257 & 16257 \\
        \bottomrule
    \end{tabular}
    \newline
    \textit{Note: This table presents the parameter estimates for various models. 
    The baseline model includes the intercept, Size, and MB. 
    The interaction model adds an interaction term between MB and D\_RD. 
    The augmented model includes additional interaction terms. 
    The RD model includes only firms with R\&D, while the No RD model includes firms without R\&D. 
    The RD Rank model includes the RD\_rank variable, 
    and the RD Class model includes categorical variables for RD\_rank. 
    Standard errors are in parentheses. 
    Significance levels are indicated by * p<0.1, ** p<0.05, *** p<0.01.}
\end{table}

\end{document}